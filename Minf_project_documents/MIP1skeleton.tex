% Do not change the options here
\documentclass[bsc,frontabs,singlespacing,parskip,deptreport]{infthesis}

\begin{document}
\begin{preliminary}

\title{Implementing a gather.town clone in Links}

\author{Weston Everett}

\course{Master of Informatics}
\project{{\bf MInf Project (Part 1) Report}}

\date{\today}

\abstract{
This skeleton demonstrates how to use the \texttt{infthesis} style for
undergraduate dissertations in the School of Informatics. It also emphasises the
page limit, and that you must not deviate from the required style.
The file \texttt{skeleton.tex} generates this document and can be used as a
starting point for your thesis. The abstract should summarise your report and
fit in the space on the first page.
}

\maketitle

\section*{Acknowledgements}
Acknowledgements go here.

\tableofcontents
\end{preliminary}


\chapter{Introduction}

The preliminary material of your report should contain:
\begin{itemize}
\item
The title page.
\item
An abstract page.
\item
Optionally an acknowledgements page.
\item
The table of contents.
\end{itemize}

As in this example \texttt{skeleton.tex}, the above material should be
included between:
\begin{verbatim}
\begin{preliminary}
    ...
\end{preliminary}
\end{verbatim}
This style file uses roman numeral page numbers for the preliminary material.

The main content of the dissertation, starting with the first chapter,
starts with page~1. \emph{\textbf{The main content must not go beyond page~40.}}

The report then contains a bibliography and any appendices, which may go beyond
page~40. The appendices are only for any supporting material that's important to
go on record. However, you cannot assume markers of dissertations will read them.

You may not change the dissertation format (e.g., reduce the font size, change
the margins, or reduce the line spacing from the default 1.5 spacing). Be
careful if you copy-paste packages into your document preamble from elsewhere.
Some \LaTeX{} packages such as \texttt{geometry}, \texttt{fullpage}, or
\texttt{savetrees} change the margins of your document. Do not include them!

Over length or incorrectly-formatted dissertations will not be accepted and you
would have to modify your dissertation and resubmit. You cannot assume we will
check your submission before the final deadline and if it requires resubmission
after the deadline to conform to the page and style requirements you will be
subject to the usual late penalties based on your final submission time.

\section{Using Sections}

Divide your chapters into sub-parts as appropriate.

\section{Citations}

Citations (such as \cite{P1} or \cite{P2}) can be generated using
\texttt{BibTeX}. For more advanced usage, the \texttt{natbib} package is
recommended. You could also consider the newer \texttt{biblatex} system.

These examples use a numerical citation style. You may also use
(Author, Date) format if you prefer.

\chapter{Background Information}

(introduction will of course cover that the project is a gather.town clone in Links)

\section{gather.town (and clones)}

Gather.town is a website which allows a user to move around a map and speak to other users
in the same "room", automatically beginning voice/video calls with anyone in their close proximity.
It also includes other features such as areas where the range of your calls is changed
(such as a table where you are only talking to other people at the table or a podium where everyone in the room can hear you),
objects that allow you to load documents for others to see, and a space/avatar designer for users to customize their avatars and surroundings.

 In theory, this allows workplaces and events to emulate the feel of an in-person meeting and allows
 a more casual way to have conversations at these sort of events as opposed to planning a group call in advance.

 In practice, although many of these benefits are realized and the style of platform is (subjectively) better for (certain types of events?)
 users also have complaints about the system (insert citation here).  For example, the sudden and uncontrolled entering of conversations with surrounding people

\section{Links}

\section{WebRTC}



In order to

A dissertation usually contains several chapters.

\chapter{Conclusions}

\section{Final Reminder}

The body of your dissertation, before the references and any appendices,
\emph{must} finish by page~40. The introduction, after preliminary material,
should have started on page~1.

You may not change the dissertation format (e.g., reduce the font size, change
the margins, or reduce the line spacing from the default 1.5 spacing). Be
careful if you copy-paste packages into your document preamble from elsewhere.
Some \LaTeX{} packages such as \texttt{geometry}, \texttt{fullpage}, or
\texttt{savetrees} change the margins of your document. Do not include them!

Over length or incorrectly-formatted dissertations will not be accepted and you
would have to modify your dissertation and resubmit. You cannot assume we will
check your submission before the final deadline and if it requires resubmission
after the deadline to conform to the page and style requirements you will be
subject to the usual late penalties based on your final submission time.

\bibliographystyle{plain}
\bibliography{mybibfile}

%% You can include appendices like this:
% \appendix
%
% \chapter{First appendix}
%
% \section{First section}
%
% Markers do not have to consider appendices. Make sure that your contributions
% are made clear in the main body of the dissertation (within the page limit).

\end{document}
